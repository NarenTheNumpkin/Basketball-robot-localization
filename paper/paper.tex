\documentclass[a4paper]{article}

\usepackage{geometry}
\geometry{left=1.5cm, right=1.5cm, top=2.54cm, bottom=2.54cm}
\usepackage{graphicx, hyperref, setspace, amsmath, amssymb, titlesec, fancyhdr, multicol, parskip, indentfirst, etoolbox, caption, cite, xcolor}
\usepackage{algorithm}
\usepackage{algpseudocode}
\usepackage{cuted}

\titleformat{\section}{\centering\large\scshape}{\thesection}{1em}{}
\titleformat{\subsection}{\normalsize\bfseries}{\thesubsection.}{1em}{}
\setstretch{1.0} 
\setlength{\parskip}{6pt} 
\titlespacing{\section}{0pt}{6pt}{6pt}
\titlespacing{\subsection}{0pt}{6pt}{6pt}
\titlespacing{\subsubsection}{0pt}{6pt}{6pt}

\title{Real-Time Localization Framework for Autonomous Basketball Robots}

\date{} 
\captionsetup{labelfont={small,sc}, textfont={small,sc}}

\renewcommand{\thesection}{\Roman{section}.}
\renewcommand{\thesubsection}{\textit{\Alph{subsection}.}}
\renewcommand{\thesubsubsection}{\textit{\arabic{subsubsection}.}}
\renewcommand{\thetable}{\Roman{table}} 
\renewcommand{\thefigure}{\Roman{figure}} 

\titleformat{\subsection}{\normalfont\large\itshape}{\thesubsection}{1em}{}
\titleformat{\subsubsection}{\normalfont\itshape}{\thesubsubsection}{1em}{}


\makeatletter
\renewcommand{\maketitle}{
  \begin{center}
    \noindent\rule{\textwidth}{1pt}\par
    \vspace{0.3em}
    {\LARGE \bfseries \@title \par}
    \vspace{0.3em}
    \noindent\rule{\textwidth}{1pt}\par
  \end{center}
}
\makeatother

\begin{document}

\maketitle
\vspace{1cm}


\begin{multicols}{2}
    \centering
    \textbf{Naren Medarametla}\\
    \textit{School of Computer Science Engineering}\\
    \textit{Vellore Institute of Technology}\\
    \textit{Chennai, India}\\
    \texttt{naren.medarametla2023@vitstudent.ac.in}
    \vfill\null

    \columnbreak

    \textbf{Sreejon Mondal}\\
    \textit{School of Electrical and Electronics Engineering}\\
    \textit{Vellore Institute of Technology}\\
    \textit{Chennai, India}\\
    \texttt{sreejon.mondal2023@vitstudent.ac.in}
    \vfill\null
\end{multicols}

\singlespacing
\setlength{\parskip}{6pt}
\setlength{\parindent}{0.5cm}

\begin{multicols}{2}
\setlength{\columnsep}{0.5cm}

\noindent \textbf{\textit{Abstract---}
Localization is a fundamental capability for autonomous robots, enabling them 
to operate effectively in dynamic environments. In Robocon 2025, accurate and 
reliable localization is crucial for improving shooting precision, avoiding 
collisions with other robots, and navigating the competition field efficiently. 
In this paper, we propose a hybrid localization algorithm that integrates classical 
techniques with learning-based methods, relying solely on visual data from the court’s
floor to achieve self-localization on the basketball field.
}

\small	
\noindent \textbf{
  \textit{Keywords---}\textit{Robot Localization, Autonomous Navigation, Neural Networks, Robocon}}

\section{Introduction}
\par \noindent
\textbf{Section II} reviews existing methods and prior research related to this work. 
\textbf{Section III} provides a detailed description of our proposed algorithm, approach, and model architecture. 
\textbf{Section IV} provides the results obtained from our experiments. 
\textbf{Section V} evaluates the accuracy of our approach and discusses potential directions for future work.


\section{Related Work}

\section{Methodology}
\par \noindent Our approach is a two-step process that begins with \textbf{Preprocessing} the image,
followed by passing it to the model for \textbf{Inference}.
\subsection{Preprocessing}
\par \noindent The input image having dimensions (640 $\times$ 480 $\times$ 3) is converted from the RGB color space to the HSV color space, then the white regions are masked out using two predefined HSV ranges.
The image is downsampled through a radial scan, flattened, and finally passed through the neural network.
Figure I shows the preprocessing pipline and Algorithm 1 gives the implementation of the downsampling algorithm.

\par \noindent

{ \centering
  \includegraphics[width=0.5\textwidth]{../results/Flowchart.png}\\
  \captionof{figure}{preprocessing pipeline}\label{fig:flowchart}
}
\begin{algorithm}[H]
  \caption{Downsampling}
\begin{algorithmic}[1]
\Statex \textbf{Input: }Image
\State $H \gets$ Image.height
\State $W \gets$ Image.width
\State $R \gets$ Black Image
\For{angle $\gets$ 0 to 180 step 2}
    \State $lastPixel \gets 0$
    \State $cx \gets \text{W / 2}$
    \State $cy \gets \text{H}$
    \For{d $\gets$ 0 to max(H, W)}
        \State $x \gets cx + d \times \cos(angle)$
        \State $y \gets cy - d \times \sin(angle)$
        \If{$0 \leq x < \text{W} \ \textbf{and} \ 0 \leq y < \text{H}$}
          \State $pixel \gets Image[y][x]$
          \If{lastPixel = 255 and pixel $\neq$ 255}
            \State $R[y][x] \gets 255 $ 
          \EndIf
          \State $lastPixel \gets pixel$
        \EndIf
    \EndFor
\EndFor
\State \textbf{Return} R
\end{algorithmic}
\end{algorithm}

\subsection{Model Architecture}
\par \noindent
The proposed model is a feedforward neural network consisting of a flattening layer followed 
by four fully connected layers. The first 200 pixels from the top of the image are removed 
to reduce the input size, as they do not carry useful information and the pixel values are scaled
down to the range [0, 1].

\par \noindent
The resulting image with dimensions (640 $\times$ 280 $\times$ 1) is flattened into a vector 
of size 1,79,200 and passed through the first linear layer, followed by a ReLU activation function.
The subsequent layers have sizes 1024, 256, and 64, each followed by a ReLU activation. 
The final layer outputs a 2-dimensional vector representing the predicted $x$ and $y$ positions of the robot.

\par \noindent
The rationale for using this relatively simple model lies in the simplicity of the input images 
and to reducing inference time.

%maybe could add input dimensions for each layer??
{ \centering
 \includegraphics[width=0.2\columnwidth]{../results/model.png}\\
 \captionof{figure}{architecture diagram}\label{pinki}
}

\subsection{Dataset} 

\par \noindent
A digital twin of the robot was created and simulated in a replica of the Robocon 2025 arena 
using the Gazebo \cite{gazebo} simulator with the help of ROS 2 \cite{macenski2022ros2}. 

\par \noindent
The robot was driven through the arena capturing images and corresponding $x$, $y$ coordinates. A 
TimeSynchronizer was used to synchronize the frame header and the position header. A total of
6283 images were captured and split into training  and test datasets with a 0.9 to 0.1 ratio.
Care was taken to include every part of the arena for an unbiased dataset. 

% maybe could get a better photo?
{ \centering
 \includegraphics[scale=0.3]{../results/gazebo.png}\\
 \captionof{figure}{gazebo simulation}\label{pinki}
}

\subsection{Training}
\par \noindent
The model was trained for 15 epochs using an Adam optimizer \cite{kingma2014adam} with an initial learning rate of $10^{-4}$
and a Mean Squared Error (MSE) loss function in batches of size 8. 

\section{Results}
% add repo link as well
\par \noindent
Figure IV depicts the loss at each epoch throughout the training process.
8 independent images at different points of the court were again captured 
from the simulation and fed into the model, Figure V shows the plot between
the ground truth and the prediction made.

{ \centering
 \includegraphics[scale=0.3]{../results/loss-epochs.png}\\
 \captionof{figure}{loss curve}\label{pinki}
}

{ \centering
 \includegraphics[scale=0.2]{../results/comparison.png}\\
 \captionof{figure}{truth vs prediction}\label{pinki}
}


\section{Conclusion}  % write about future work as well
% maybe include other features such as IMU data, odometry

\begin{thebibliography}{99}

\bibitem{gazebo}
N. Koenig and A. Howard, "Design and use paradigms for Gazebo, an open-source 
multi-robot simulator," 2004 IEEE/RSJ International Conference on Intelligent Robots 
and Systems (IROS) (IEEE Cat. No.04CH37566), Sendai, Japan, 2004, pp. 2149-2154 vol.3, 
doi: 10.1109/IROS.2004.1389727.
  
\bibitem{macenski2022ros2}
Steve Macenski, Tully Foote, Brian Gerkey, Chris Lalancette, and William Woodall. 
\textit{Robot Operating System 2: Design, architecture, and uses in the wild}. 
Science Robotics, vol. 7, no. 66, 2022. 

\bibitem{kingma2014adam}
Diederik P. Kingma and Jimmy Ba. 
\textit{Adam: A method for stochastic optimization}. 
arXiv preprint arXiv:1412.6980, 2014.

\end{thebibliography}

\end{multicols}

\end{document}