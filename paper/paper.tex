\documentclass[a4paper]{article}

\usepackage{geometry}
\geometry{left=1.5cm, right=1.5cm, top=2.54cm, bottom=2.54cm}
\usepackage{graphicx, hyperref, setspace, amsmath, amssymb, titlesec, fancyhdr, multicol, parskip, indentfirst, etoolbox, caption, cite, xcolor}

\titleformat{\section}{\centering\large\scshape}{\thesection}{1em}{}
\titleformat{\subsection}{\normalsize\bfseries}{\thesubsection.}{1em}{}
\setstretch{1.0} 
\setlength{\parskip}{6pt} 
\titlespacing{\section}{0pt}{6pt}{6pt}
\titlespacing{\subsection}{0pt}{6pt}{6pt}
\titlespacing{\subsubsection}{0pt}{6pt}{6pt}

\title{Vision-Based Localization for Autonomous Basketball Robots}

\date{} 
\captionsetup{labelfont={small,sc}, textfont={small,sc}}

\renewcommand{\thesection}{\Roman{section}.}
\renewcommand{\thesubsection}{\textit{\Alph{subsection}.}}
\renewcommand{\thesubsubsection}{\textit{\arabic{subsubsection}.}}
\renewcommand{\thetable}{\Roman{table}} 
\renewcommand{\thefigure}{\Roman{figure}} 

\titleformat{\subsection}{\normalfont\large\itshape}{\thesubsection}{1em}{}
\titleformat{\subsubsection}{\normalfont\itshape}{\thesubsubsection}{1em}{}


\makeatletter
\renewcommand{\maketitle}{
  \begin{center}
    \noindent\rule{\textwidth}{1pt}\par
    \vspace{0.3em}
    {\LARGE \bfseries \@title \par}
    \vspace{0.3em}
    \noindent\rule{\textwidth}{1pt}\par
  \end{center}
}
\makeatother

\begin{document}

\maketitle
\vspace{1cm}


\begin{multicols}{2}
    \centering
    \textbf{Naren Medarametla}\\
    \textit{School of Computer Science Engineering}\\
    \textit{Vellore Institute of Technology}\\
    \textit{Chennai, India}\\
    \texttt{naren.medarametla2023@vitstudent.ac.in}
    \vfill\null

    \columnbreak

    \textbf{Sreejon Mondal}\\
    \textit{School of Electrical and Electronics Engineering}\\
    \textit{Vellore Institute of Technology}\\
    \textit{Chennai, India}\\
    \texttt{sreejon.mondal2023@vitstudent.ac.in}
    \vfill\null
\end{multicols}

\singlespacing
\setlength{\parskip}{6pt}
\setlength{\parindent}{0.5cm}

\begin{multicols}{2}
\setlength{\columnsep}{0.5cm}

\noindent \textbf{\textit{Abstract---}
Localization is a fundamental capability for autonomous robots, enabling them 
to operate effectively in dynamic environments. In Robocon 2025, accurate and 
reliable localization is crucial for improving shooting precision, avoiding 
collisions with other robots, and navigating the competition field efficiently. 
In this paper, we propose a hybrid localization algorithm that integrates classical 
techniques with learning-based methods, relying solely on visual data from the court’s
floor to achieve self-localization on the basketball field.
}

\small	
\noindent \textbf{
  \textit{Keywords---}\textit{Robot Localization, Autonomous Navigation, Neural Networks, Robocon}}

\section{Introduction}
\textbf{Section II} reviews existing methods and prior research related to this work. 
\textbf{Section III} provides a detailed description of our proposed algorithm, approach, and model architecture. 
\textbf{Section IV} provides the results obtained from our experiments. 
\textbf{Section V} evaluates the accuracy of our approach and discusses potential directions for future work.

\section{Related Work}
\section{Methodology}


\section{Results and Analysis}

\section{Conclusion and Future Work}

\section{References}

\end{multicols}

\end{document}